%===============================================================================
\chapter{Úvod}

Dnešní doba poskytuje mnoho možností, jak si svůj vyvíjený produkt, nebo například jen pouhou hypotézu toho, jak by nějaký systém mohl fungovat, namodelovat a provést na tomto modelu simulaci. Z~takto provedené simulace získáme představu o~tom, jak se náš produkt chová. A~pro tyto simulace je potřeba zvolit vhodný simulační nástroj. Vhodně zvolený simulační nástroj může totiž ušetřit nemalé prostředky při uvedení produktu na trh právě díky proběhnutí simulací a odlazením celého procesu chodu produktu. Proto je vhodné věnovat čas simulaci dostatečný čas. Simulační nástroje jsou implementovány nad různými simulačními jazyky, ale pro účely práce byly zvoleny primárně dvě hlavní rodiny jazyků - MATLAB a Modelica, případně jazyky jim podobné. Zároveň je pro porovnání vybrán jazyk Python s~rozšířeními Scipy a Numpy, který není příbuzný s~výše uvedenými jazyky.

Cílem této bakalářské práce je provést srovnání volně dostupných simulačních nástrojů. Pro účely porovnání budou implementovány simulační modely v~různých simulačních jazycích. Tyto modely budou následně porovnány pomocí porovnávací metodiky a získané výsledky srovnání budou použity k~pojednání o~vhodnosti použití implementovaných modelů pro zvolené nástroje.

V~kapitole \ref{kapitola2} budou popsány a vysvětleny důležité pojmy potřebné pro pochopení problematiky modelování a simulací. Zároveň v~této kapitole bude vysvětlena problematiky porovnání a popsána zvolená porovnávací metodika. Dále bude kapitola pojednávat o~použitých simulačních nástrojích a jejich jazicích. Kapitola \ref{kapitola4} bude obsahovat popis implementovaných simulačních modelů pro simulační nástroje. V~následující kapitole \ref{kapitola5} bude popsán průběh porovnání nástrojů. Poslední kapitola \ref{kapitola6} bude obsahovat výsledky porovnání simulačních nástrojů zároveň s~popisem způsobilosti použitých modelů pro využité nástroje.

\chapter{Modelování a simulace}
\label{kapitola2}

Modelování a simulace jsou dnes již nedílnou součástí vědy a výzkumu. V~této kapitole se věnuji popisu pojmů nutných k~pochopení základní problematiky modelování. Dále je zde možné najít popis problematiky porovnávání a porovnávací metodiku.

\section{Problematika modelování a simulací}
Jelikož jsou modelování a simulace velmi komplexní disciplíny, je nutné si stanovit, alespoň v~jednoduchosti, základní pojmy, které jsou v~rámci této práce využívány.

\subsection{Modelování a model}
\label{modelovani}
Modelování je vědecká disciplína, při které dochází k~popisu systému většinou z~reálného světa a získání všech možných informací, které k~tomuto systému máme. Ze získaných poznatků se vytvoří model. V~rámci zkoumání získaných informací je nutná analýza možných nepochopitelných a nepopsatelných vlastností modelu. Model je tedy jen přiblížení se ke skutečnosti, nikoliv její dokonalý odraz. Modely lze dělit podle jejich chování na:

\begin{itemize}
    \item spojité -- hodnoty modelu se v~průběhu simulace mění spojitě. Tyto modely jsou popsatelné diferenciálními rovnicemi
    \item diskrétní -- hodnoty modelu se v~průběhu simulace mění skokově. Tyto modely lze popsat jako konečný, nebo celulární automat
    \item kombinované -- tyto modely obsahují společně spojité a diskrétní složky vedle sebe
\end{itemize}

V~literatuře můžeme nalézt další možné rozdělení modelů \cite{IMS-skripta}.
\newline

S~takto popsanými modely je pak možné vykonávat simulace. Během simulace se pozoruje chování modelu, vyhodnocují se výsledky simulací a případně se mění popis modelu, pokud to vyžaduje simulace, nebo pokud se přijde na nové, nečekané chování.

Model je také možné vytvářet pomocí tzv. "nereálného"\space systému. Takové systémy jsou často k nalezení v počítačových hrách, případně ve vědeckých oborech, kde jsou získané poznatky o systému velmi omezené.

\subsection{Systém}
Systém je možno definovat jako soustavu částic, které na sebe vzájemně působí. Systémy se mohou dělit na:

\begin{itemize}
    \item Otevřené a uzavřené -- podle jejich interakce s~okolním prostředím
    \item Spojité a diskrétní -- podle toho, zda se hodnoty systému mění spojitě, nebo skokově
    \item Statické a dynamické -- podle toho, zda se systém vyvíjí v~čase
\end{itemize}
Pro potřeby simulací jsou využívány především dynamické systémy, u~kterých můžeme sledovat jejich vývoj v~průbehu simulace.

\subsection{Simulace}
Simulace je disciplína zabývající se získáváním informací z~modelů pomocí provádení experimentů s~nimi. Pro získání validních informací o~modelu je potřeba provést simulaci vícekrát a následně porovnat a vyhodnotit výsledné hodnoty.

Před během simulace a po vyhodnocení výsledků je vhodné provést analýzu vhodnosti modelu pro simulaci. Pokud shledáme model nevalidním, není vhodný pro náš záměr a je záhodno jej z~dalších simulací vyřadit.


Simulace můžeme rozdělovat podle mnoha kritérii:
\begin{enumerate}
    \item Podle typu použitého modelu:
        \begin{itemize}
            \item Spojitá -- rozdělení je závislé podle typu použitého modelu, viz \ref{modelovani}
            \item Diskrétní
            \item Kombinovaná
        \end{itemize}
    \item Podle zpracování výsledků:
        \begin{itemize}
            \item Kvalitativní -- v~rámci kvalitativní metody nevnímáme model jako komplexní objekt, ale zaměřujeme se pouze na pár vybraných kvalit
            \item Kvantitativní -- tato metoda zpracování je vhodná pro modely, které neumíme přesně zapsat a popsat
        \end{itemize}
    \item Podle rozložení výpočtů:
        \begin{itemize}
            \item Simulace na jednom stroji -- pro účely výpočtu je využit jen jeden výpočetní stroj (procesor, cluster). Toto využití je vhodné pro simulace jednoduššího rázu.
            \item Paralelní a distribuovaná simulace -- simulace probíhá na více strojích zároveň. Pomocí dilčího rozdělení výpočtů můžeme dosáhnout značného zrychlení v~případě složitých simulací, např. lety do kosmu.
        \end{itemize}
\end{enumerate}
Další a mnohá rozdělení simulací můžeme nalézt v~literatuře. Pro účely různých druhů simulací jsou rozdělení jiná, nemluvě o~dynamickém vývoji tohoto odvětví a tudíž i neustálým změnám v~rozdělení simulací.

\section{Postup modelování a simulace}
Jak už bylo řečeno v kapitole \ref{modelovani}, modely jsou tvořeny pomocí pozorování systému a získávání poznatků o něm. Pro účely popisu tvoření modelu budeme vycházet z předpokladu, že pozorujeme tzv. "reálný"\space systém. Na základě získaných reálií o systému vytvoříme abstraktní model, který je vlastně zjednodušenina pozorovaného systému, díky tomu, že pracujeme pouze s daty, která získáme pomocí pozorování. Takto vytvořený model může být popsán například diferenciálními rovnicemi.

Z takto popsaného abstraktního modelu dále vytvoříme simulační model. Simulační model je vlastně přepis abstraktního modelu do programovacího jazyka tak, abychom byli schopni s tímto modelem provádět měření a experimenty. Simulační model je tedy program/skript, který počítá výsledky podle zadaných vstupních hodnot, počátečního stavu, parametrů modelu a případných proměnných.

S výsledky získánými ze simulačního modelu získáváme další poznatky o chování abstraktního modelu. Tyto získané výsledky můžeme využít pro upravení abstraktního modelu, následně upravení simulačního modelu podle modelu abstraktního a opětovnému provedení simulace \cite{IMS-skripta}.

Výsledky je vhodné vizualizovat tak, abychom mohli s nimi dále pracovat a analyzovat je. V rámci analýzy je potřeba rozlišit s jakým typem dat pracuji a podle toho zvolit metodu, pomocí které budu analýzu provádět. Mezi běžné druhy analýzy výsledků patří:
\begin{itemize}
    \item Porovnání získaných výsledků s reálnými naměřenými daty daného systému
    \item Použití statistických metod pro zpracování výsledků
    \item Automatické vyhodnocení výsledků simulace podle metodiky zvoleného simulačního nástroje
\end{itemize}

V rámci simulace můžeme sledovat i hodnoty, které nepřímo souvisí s výsledkem simulace. Mezi tyto hodnoty patří např.:
\begin{itemize}
    \item Strojový čas
    \item Přesnost použité numerické metody
    \item Alokace zdrojů při simulaci
\end{itemize}

Část z těchto proměnných budeme sledovat v rámci této bakalářské práce. Popis sledování a následného porovnání lze nalézt v kapitole \ref{kapitola5}.

\section{Výhody a nevýhody využití simulací}

Jako všechny procesy, tak i simulace mají své pozitivní i negativní vlastnosti. V jednoduchosti se dá říct, že simulace šetří čas a peníze, ale na druhou stranu mohou stát mnoho zdrojů (peníze a čas) pro to, aby vůbec byly vytvořeny a uskutečněny podle reálného podkladu.

\subsection{Výhody využití simulací}
\label{vyhody}
Jak už bylo zmíněno výše, hlavními výhodami jsou ušetření času a ceny experimentu. Dále to pak mohou být bezpečnost, replikovatelnost a rychlost. Bližší popis výhod:

\begin{enumerate}
    \item Cena je v dnešní době velmi důležitým faktorem pro potřeby simulací, protože experimenty s reálným systémem mohou být velmi drahé. Díky využití modelů a simulací je možné výrazně ušetřit na výrobě testovacích kusů. Toto je velmi patrné například v leteckém průmyslu - reálné systémy (letadla, jejich komponenty) se zde využívají až pro validaci výsledků předchozích simulací. 
    \item Rychlost je dalším z kritérií pro běh simulací. V závislosti na výkonu výkonu stroje, který simulaci vykonává, můžeme simulaci zpomalovat a zrychlovat dle našich potřeb. Důležitým faktorem je zde výkon výpočetního zdroje, který nám omezuje maximální (minimální) rychlost simulace. Dobrým příkladem pro nutnost zrychlování simulace je sledování pohybu litosferických desek. Tento pohyb je v reálném systému velmi pomalý, ale pokud máme dostatečně výkonný výpočetní stroj, můžeme simulaci zrychlit a dosáhnout výsledků o mnoho dříve. 
    \item Replikovatelnost je faktor, který velmi úzce souvisí s cenou simulace. Pomocí simulace můžeme provést výpočty na jednom modelu prakticky nekonečně mnohokrát bez potřeby využití reálného systému. Pomocí replikovatelnosti jsme schopni dosáhnout přesnějších výsledků a tím pádem i větši znalosti vlastností objektů.
    \item Bezpečnost je další faktor, který je velkou výhodou využití simulací. Některé experimenty s reálnými systémy není možné z bezpečnostních důvodů vůbec provádět - například crash testy letadel, nebo simulace chemických a jaderných reakcí, případně simulace epidemií různých nemocí.
    \item V neposlední řadě je to možnost experimentovat se systémy, které můžeme poznat právě jen díky provádění simulací s nimi - například kolize vesmírných těles, tvorba planet, působení černých děr, apod.
\end{enumerate}

Tento výpis výhod simulací rozhodně není konečný. Je velmi pravděpodobné, že se v blízké době objeví další velká výhoda tohoto procesu. Jelikož výpočetní výkon počítačů neustále roste podle Moorova zákona \cite{Schaller1997}, časem bude možné provádět experimenty s daleko složitějšími modely než doposud. 

\subsection{Nevýhody využití simulací}

Stejně tak, jako simulace mohou šetřit naše zdroje, při nesprávném vymodelování modelu, případně nesprávném zavedení simulací, se mohou velmi prodražit a prodloužit časy simulací. Se simulacemi se samozřejmě pojí více problémů, než jen peníze a čas:

\begin{enumerate}
    \item Velké nároky na výpočetní výkon je běžný problém všech výpočetních operací. S~rostoucími nároky na rychlost a výsledky simulací přímo uměrně roste také nárok na výkon simulačního stroje. Nicméně díky zvětšování výkonu v závislosti na Moorovu zákonu, viz kapitola \ref{vyhody}, je možné provádět složitější výpočetní operace i na běžných osobních počítačích a ne pouze na sálových superpočítačích, jako tomu bylo dříve.
    \item Během simulace může docházet k chybám kvůli nevhodně zvolené numerické metodě, případně kvůli její nepřesnosti řešení. Tento stav dokáže ovlivnit i relativně bezchybně namodelovaný model a znehodnotit tak výsledky z experimentů s ním. 
    \item Velká časová náročnost v případě, že chceme simulaci provést vícekrát. Tento problém je spjatý i s nároky na výpočetní výkon -- je-li výkon dostatečný, můžeme simulaci urychlit a tím i zkrátit čas simulace.
    \item Velmi důležitým problémem je problém ověřování validity. Pokud máme chybně namodelovaný model, pak výsledky simulace obsahují také chyby a tedy model není validní. Ověřování validity modelu by mělo proběhnout ještě před provedením simulace, tak, abychom předešli jeho následnému přemodelování.
\end{enumerate}

Stejně jako u výhod simulací v kapitole \ref{vyhody} není tento výčet konečný. Další velkou nevýhodou je například vysoká složitost vytvoření modelu. A stejně jako u výhod, i zde je pravděpodobné, že se další velké nevýhody budou objevovat s přibývajícím časem. 

\chapter{Popis použitých simulačních jazyků a nástrojů}
\label{kapitola3}

\section{Použité simulační jazyky}
Jazyků umožňujících vytváření a specifikaci modelů a provádení experimentů s~nimi je na trhu nepřeberné množství \cite{list-of-process-modelling-lang}. Tyto jazyky se také liší v~případech využití - pro různé typy modelů a simulací systémů jsou jinak vhodné. Pro potřeby bakalářské práce jsem si zvolil tři programovací jazyky, z~toho dva z~kategorie simulačních jazyků, které jsou využívány vybranými volně dostupnými simulačními nástroji. Nástroje tyto jazyky používají buď v~čisté formě, případně jejich modifikované verze s~použitím podpůrných knihoven. Těmito jazyky jsou MATLAB a Modelica. Jako třetí programovací jazyk byl pro porovnání zvolen jazyk Python s~podporou matematických, vědeckých a simulačních knihoven a knihoven pro tvorbu grafů~\footnote{https://scipy.org/}.

\subsection{MATLAB}
Jazyk MATLAB je uzavřený skriptovací jazyk, který je vyvíjený a spravovaný firmou MathWorks od roku 1984. Tento jazyk je spjatý se stejnojmenným simulačním nástrojem MATLAB. Jazyk je využíván k popisu analýzy dat, k návrhu a implementaci algoritmů a také k vytváření modelů a provádění operací (simulací) s nimi. Jazyk MATLAB se vyznačuje velmi slabou typovou kontrolu a také tím, že se všemi objekty je pracováno jako s maticemu, v případě jednořádkové matice, vektoru, je s práce s touto maticí shodná s prací s polem hodnot. Většina vestavěných funkcí je pro tento účel naprogramována a očekává na svém vstupu právě matici. Syntax zápisu kódu v MATLABu připomíná kterýkoliv jiný funkcionální jazyk. Jazyk MATLAB podporuje také objektově orientované programování. Pro účely zpracování a analýzy získaných výsledků dává jazyk možnost vykreslit grafy. Všechny výsledky simulací v této práci, které jsou provedeny v MATLABu, mají vykreslené grafy právě tímto nástrojem. Výsledky můžete nalézt v kapitole \ref{kapitola6}. Prostředí MATLABu je navrženo tak, že navíc dokáže pracovat s funkcemi napsanými v jazyce C a Fortran. Zároveň má jazyk možnost využít knihovny napsané v jiných jazycích, jako jsou například Java, .NET nebo Perl.
\begin{figure}
    \centering
    \includegraphics[width=15cm]{obrazky-figures/dummy.png}
    \caption{Ukázka prostředí MATLAB}
    \label{fig:matlab}
\end{figure}

Tento programovací jazyk byl vybrán kvůli rozšířenosti na středních a vysokých školách, ale také na různých pracovištích zabývajícími se modelováním a simulacemi. Navzdory tomu, že nástroj MATLAB není volně dostupný, na trhu jsou k dostání volně dostupné nástroje, jako například GNU Octave a SciLab, které jsou s jazykem MATLAB kompatibilní. Dále pro MATLAB existuje rozšíření SIMULINK, které do prostředí MATLABu přidává možnost provádět vícerozměrné grafické simulace. Také tyto nástroje jsou ve výběru nástrojů využitých v této bakalářské práci.

Jazyk MATLAB bohužel není volně dostupný, nicméně v akademickém prostředí je velmi běžně k dostání, stejně jako v následné praxi. Pro účely bakalářské práce je využita studentská licence ve vlastnictví VUT FIT.


\subsection{Modelica}
Jazyk Modelica je objektově orientovaný, modelační a simulační jazyk, který je vyvíjen neziskovou organizací Modelca Association od roku 1997. Díky tomu, že je tento jazyk volně dostupný, je na trhu mnoho simulačních nástrojí a vývojových prostředí, které pracují právě s tímto jazykem. Pro účely této práce byly vybrány nástoje Dymola, OpenModelica a jModelica.org. Vývojové prostředí Dymola není, stejně jako MATLAB, volně dostupné. Pro účely provádění experimentů s tímto vývojovým prostředím je využita licence, pomocí které je možno spustit nástroj Dymola na školním serveru Merlin v rámci VUT FIT.

Modely, které jsou zapsány v jazyku Modelica, mají veškeré své chování popsané rovnicemi. Například rovnice Newtonova zákona ochlazování, který má matematický zápis:

$$
mc_p\Dot{T} = hA(T_{\infty} - T)
$$

se v jazyce Modelica zapíše naprosto totožným způsobem. Rovnice má tedy po zapsání následující tvar:

\begin{verbatim}
                        m*c_p*der(T) = h*A*(T_inf-T)
\end{verbatim}

Pro to, aby byl model funkční, je nutné definovat vstupní hodnoty jednotlivých proměnných. Zapsaná rovnice se tedy chová naprosto stejně, jako když počítáme rovnici na papíře. Použítím znaku $ = $ tedy nedochází k přiřazení, jako je to běžné u jiných programovacích jazyků, ale slouží zde jako znak pro porovnání - proto tedy ona práce s rovnicemi.

\begin{figure}
    \centering
    \includegraphics[width=15cm]{obrazky-figures/dummy.png}
    \caption{Ukázka prostředí jModelica.org}
    \label{fig:modelica}
\end{figure}

Jazyk je v současnosti velmi využíván v oblasti proudění kapalin a termodynamiky, vzhledem k jednoduchému vytvoření modelu podle předem stanovených rovnic. Dále se pak používá ve větší míře v elektrotechnice a strojírenství.

Stejně jako jazyk MATLAB, i Modelica podporuje grafický přehled výsledků pomocí grafů. I v tomto případě jsou v kapitole \ref{kapitola6} grafy vytvořeny pomocí vestavěného nástroje pro tvorbu grafů v jazyce Modelica.


\subsection{Python}
Python je volně dostupný interpretovaný programovací jazyk, který má nepřeberné množství využití. Poprvé byl představen v roce 1991. Jazyk Python není v základním pojetí simulačním nebo modelačním jazykem. Tuto funkcionalitu získává teprve s dodatečnými knihovnami. V rámci využití v této bakalářské práci byla vybrána rodina knihoven SciPy. Hlavním přínosem této rodiny knihoven je možnost použití metod integrace, lineární algebry, rychlé Fourierovy transformace a řešení diferenciálních rovnic.

Tento jazyk byl zvolen z důvodu, který byl řečený výše. Python nemá primární použití jako jazyk vhodný pro modelování a simulace. Nicméně s jeho rostoucí popularitou a rostoucí uživatelskou základnou, bude jeho využití růst ještě více. Momentálně má Python velmi široký záběr pokrytí - od funkcionální, přes logické programování, až po tvorbu grafiky, databází, nebo provádění simulací. Jazyk je využíván velkými firmami na trhu, jako jsou například Google, NASA nebo Wikipedia.

Stejně jako u jazyků MATLAB a Modelica jsou grafy vykreslené v jazyce Python využity k zobrazování výsledků simulací v kapitole \ref{kapitola6}

\section{Popis použitých simulačních nástrojů}


\chapter{Implementace modelů}
\label{kapitola4}

\section{Kruhový test}

\section{Skákající míček}

\section{Tuhý systém}

\section{Diskrétní simulace}


\chapter{Porovnání}
\label{kapitola5}

\section{Základní problematika}

\section{Popis použité metodiky}


\chapter{Výsledky simulací a porovnání výsledků}
\label{kapitola6}

\chapter{Závěr}
\label{kapitola7}


%===============================================================================
